\newcommand{\deff} [2] {\uppercase{\scriptsize #1} {\tiny #2}}
% ^ this is a latex macro.
% it is called with the command `deff` and takes 2 arguments,
% \deff{name}{means this} => NAME means this 
%  ^    ^     ^ the definition
%  |    |what to define
%  |command name
\section{Definitions}

% Lecture 1

\deff{OS}{A virtual machine, resource allocator, and control program whos goal
is the comfort of the user and then efficienty of operations}

\deff{Interrupts}{A hardware signal that causes the CPU to jump to a predefined
instruction called the \textit{interrupt handler}}

\deff{Mode Bit}{Allows for dual mode operations, like \textit{user mode} and
\textit{system mode}, where there are privileged instructions, such as (en/dis)abilng
interrupts, halting CPI, writing to registers, etc.}

\deff{Process}{A program in execution. A program has potential for execution.
Contains the state of a program}

\deff{PCB}{(Process Control Blockr) Contains state, program counter, cpu
registers/scheduling info, i/o info, etc}

\deff{Context Switch}{switching the cpu to another process by saving the state
of the old one and loading the saved state for the new proccess. this happens
only when a process calls yield(), makes a syscall, or the timer decides to
switch}

\deff{Zombie}{A process that has not been cleaned up by its parent}

% =================Lecture 2==========================
\deff{System Call}{a function call that invokes the operating system}

\deff{independent}{a process that cannot affect or be affected by other process (no data sharing)}

\deff{cooperating}{a process that is not independent}

\deff{thread}{a control flow through a program}

\deff{multithreaded}{a program with multiple control flows}

